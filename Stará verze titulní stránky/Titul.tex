\documentclass[14pt,a4paper,oneside]{memoir}
\usepackage[utf8]{inputenc}
\usepackage[czech]{babel}
\usepackage[T1]{fontenc}
\usepackage{amsmath}
\usepackage{amsfonts}
\usepackage{amssymb}
\usepackage{makeidx}
\usepackage{graphicx}
\usepackage[left=3.5cm,right=1.5cm,top=3cm,bottom=3.5cm]{geometry}
\frenchspacing
\pagestyle{empty}
\setlength{\parindent}{0pt}
\author{Lukáš Brtna}
\title{Titulní stránka}
\begin{document}
\begin{center}
%\begin{large}
\textbf{ZÁPADOČESKÁ UNIVERZITA V PLZNI\\
~FAKULTA ELEKTROTECHNICKÁ\\
%\vspace*{5mm}
}
\begin{small}
\textbf{~~KATEDRA TEORETICKÉ ELEKTROTECHNIKY}
\end{small}
%\end{large}
\vspace{70mm}
\begin{HUGE}
\textbf{DIPLOMOVÁ PRÁCE}
\vspace{8mm}\\
\end{HUGE}
\begin{large}
Modelování vf zařízení
\end{large}
\vspace{90mm}\\
\end{center}
\begin{flushleft}
\textbf{Plzeň 2013}
\hfill
\textbf{Lukáš BRTNA}
\end{flushleft}
\newpage
\begin{large}
\textbf{Anotace \vspace{5mm}\\}
\end{large}
Účelem diplomové práce je nastínit problematiku vf obvodů a šíření vf vln, matematický popis vf šíření a následné zapracování získaných znalostí ve formě odvozených slabých forem do xml modulu pro Agros.
\vspace{100mm}\\
\begin{large}
\textbf{Klíčová slova \vspace{5mm}\\}
\end{large}
TE vlna, TM vlna, TEM vlna, slabá forma, vf modelování
\newpage
\begin{large}
\textbf{Abstract \vspace{5mm}\\}
\end{large}
The objective of the diploma thesis is to summarize hf wave propagation and create mathematical description of hf wave propagation. The knowledge is subsequently processed in the weak forms of propagation and creation an xml modul for Agros2D.
\vspace{100mm}\\
\begin{large}
\textbf{Keywords \vspace{5mm}\\}
\end{large}
TE Wave, TM Wave, TEM Wave, Weak Form, HF Modelling
\newpage
\vspace*{50mm}
\begin{large}
\textbf{Prohlášení \vspace{5mm}\\}
\end{large}
Předkládám tímto k posouzení a obhajobě diplomovou práci, zpracovanou na závěr studia na Fakultě elektrotechnické Západočeské univerzity v Plzni.\vspace{5mm}\\
Prohlašuji, že jsem tuto diplomovou práci vypracoval samostatně, s použitím odborné literatury a pramenů uvedených v seznamu, který je součástí této diplomové práce.
\vspace{\fill}
\begin{flushleft}
V Plzni dne \today
\hspace{\fill}
Jméno a příjmení~~~~~~~~\vspace{3mm}
\end{flushleft}
\begin{flushright}
.............................................
\end{flushright}
\newpage
\vspace*{50mm}
\begin{large}
\textbf{Poděkování \vspace{5mm}\\}
\end{large}
Tímto bych rád poděkoval vedoucímu diplomové práce, panu Ing. Davidu Pánkovi, za jeho cenné rady a profesionální vedení bez nějž by vznik této práce nebyl vůbec možný.

\end{document}