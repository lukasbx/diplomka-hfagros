\documentclass[12pt,a4paper,oneside]{article}
\usepackage[utf8]{inputenc}
\usepackage[czech]{babel}
\usepackage[T1]{fontenc}
\usepackage{amsmath}
\usepackage{amsfonts}
\usepackage{amssymb}
\usepackage{makeidx}
\usepackage{graphicx}
\usepackage{enumerate}
\usepackage[font=small, labelfont=small, labelfont=bf]{caption} %Zobrazení textu "Obr. 3.1:" tučně podle normy; "font=small" písmo podle normy musí být o 2 body menší 
\usepackage[unicode=true]{hyperref} %zobrazení kapitol v pdf rejsříku + hyperaktivnost; unicode=true důležité pro správné zobrazení znaků v pdf rejstříku
\usepackage{indentfirst} %odsazení i prvního odstavce po nadpisu pro lepší strukturu
\usepackage{fancyhdr} %záhlaví a zápatí
\usepackage[left=3.5cm,right=2cm,top=3cm,bottom=3cm]{geometry}
\addto\captionsczech{\renewcommand{\figurename}{Obr.}} %V popisu obrázku zobrazí správně podle normy "Obr." místo "Obrázek"
\addto\captionsczech{\renewcommand{\tablename}{Tab.}} %V popisu tabulky zobrazí správně podle normy "Tab." místo "Tabulka"
\numberwithin{equation}{section} %Přidává k číslu rovnice číslo kapitoly (sekce)
\numberwithin{figure}{section} %Přidává k číslu obrázku číslo kapitoly (sekce)
\numberwithin{table}{section} %Přidává k číslu tabulky číslo kapitoly (sekce)
\frenchspacing %správné dělání mezer
\sloppy %dovoluje Texu dělat větší mezery a odstraňuje problém s přečnívajícími řádky
\setlength{\parindent}{5ex} %správné odsazení na pět úhozů (písmene "x")
\setlength{\parskip}{1ex} %mezera na výšku "x" mezi odstavci pro lepší čitelnost
\author{Lukáš Brtna}
\title{Diplomová práce - HF Modul pro Agros}

\newcommand{\me}{\mathrm{e}} % Eulerovo cislo
\newcommand{\mi}{\mathrm{i}} % komplexni jednotka
\newcommand{\mj}{\mathrm{j}} % komplexni jednotka
\newcommand{\dif}{\,\mathrm{d}} % diferencial
\newcommand{\mat}[1]{\mathrm{\mathbf{{#1}}}} % tenzor nebo matice
\renewcommand{\vec}[1]{\mbox{\boldmath$#1$}} % vektor
\newcommand{\faz}[1]{{\underline{#1}}} % fazor
\newcommand{\vecfaz}[1]{\mbox{\underline{\boldmath$#1$}}} % fazor vektoru
\newcommand*{\unit}[1]{\ensuremath{\mathrm{\,#1}}} % jednotky
\newcommand*{\constant}[1]{\ensuremath{\mathrm{\,#1}}} % jednotky
\newcommand{\degree}{\ensuremath{^{\circ}}} % stupne celsia
\newcommand{\set}[1]{{\mathcal{#1}}} % oznaen množin
\newcommand{\field}[1]{{\mathbb{#1}}} % oznaen těles
\renewcommand{\Re}{\mathrm{Re}}
\renewcommand{\Im}{\mathrm{Im}}
\newcommand{\tg}{\mathrm{tg}\ }
\newcommand{\grad}{\mathrm{grad}\ }
\newcommand{\curl}{\mathrm{curl}\ }
\newcommand{\rot}{\mathrm{rot}\ }
\renewcommand{\div}{\mathrm{div}\ }
\newcommand{\const}{\mathrm{const}}
\newcommand{\konst}{\mathrm{konst}}

\begin{document}
\renewcommand\refname{Použitá literatura} %Místo slova "Reference" u Bibliography
%\newcommand{\tg}{\mathop{\rm tg}\nolimits} %Přidává schopnost psát "tg" v rovnici automaticky vpřímeným písmem
\newcommand{\cotg}{\mathop{\rm cotg}\nolimits} %Přidává schopnost psát "cotg" v rovnici automaticky vpřímeným písmem
%\newcommand{\grad}{\mathop{\rm grad}\nolimits} %Přidává schopnost psát "grad" v rovnici automaticky vpřímeným písmem
%\newcommand{\rot}{\mathop{\rm rot}\nolimits} %Přidává schopnost psát "rot" v rovnici automaticky vpřímeným písmem
\newcommand{\udiv}{\mathop{\rm div}\nolimits} %Přidává schopnost psát "div" v rovnici automaticky vpřímeným písmem; zde se musí použít "\udiv", protože "\div" je již obsazeno systémem
\newcommand{\ud}{\mathrm{d}} %Přidává schopnost psát "d" v rovnici automaticky vpřímeným písmem



%%Začátek úvodní nečíslované sekce%%
\pagestyle{empty} %nečíslování stránek


%%Titulní stránka%%
\begin{titlepage} 
\begin{center}

%%Horní nadpisy%%
\begin{large}
\textbf{ZÁPADOČESKÁ UNIVERZITA V PLZNI\\
~FAKULTA ELEKTROTECHNICKÁ\\
\vspace*{5mm}}
\end{large}
\textbf{~Katedra technologií a měření}
%tex sází křivě - není zarováno přesně na střed - musím cpát mezery na dorovnání
\vspace{70mm}\\

%%Prostřední nadpis%%
\begin{Huge}
\textbf{SEMESTRÁLNÍ PRÁCE Z PŘEDMĚTU DSKE1}
\vspace{8mm}\\
\end{Huge}
\begin{LARGE}
Modelování vf zařízení \vspace{90mm}\\
\end{LARGE}
\end{center}

%%Spodní text%%
\begin{flushleft}
\begin{large}
\textbf{Plzeň 2012}
\hfill
\textbf{Lukáš BRTNA}
\end{large}
\end{flushleft}
\end{titlepage}
\newpage


%%Anotace%%
\section*{Anotace}
Účelem diplomové práce je nastínit problematiku vf obvodů a šíření vf vln, 				matematický popis vf šíření a následné zapracování získaných znalostí ve formě 			odvozených slabých forem do xml modulu pro Agros.
\vspace{100mm}\\

%%Klíčová slova%%
\section*{Klíčová slova}
TE vlna, TM vlna, TEM vlna, slabá forma, vf modelování
\newpage


%%Abstract%%
\section*{Abstract}
The objective of the diploma thesis is to summarize hf wave propagation and create 		mathematical description of hf wave propagation. The knowledge is subsequently 			processed in the weak forms of propagation and creation an xml modul for Agros2D.
\vspace{100mm}\\

%%Keywords%%
\section*{Keywords}
TE Wave, TM Wave, TEM Wave, Weak Form, HF Modelling
\newpage



%%Definice záhlaví a zápatí
\fancypagestyle{plain}{	%definice záhlaví a zápatí u plain stránek stylu fancy
  \lhead{\textit{Modelování vf zařízení}}
  \rhead{Lukáš Brtna~~~2013}
  \cfoot{- \thepage -}
}
\lhead{\textit{Modelování vf zařízení}}
\rhead{Lukáš Brtna~~~2013}
\cfoot{-~\thepage ~-}
\pagestyle{fancy}
\setcounter{page}{1}


%%Obsah%%
\setlength{\parskip}{0ex} %Aby nebyly moc velké mezery mezi řádky obsahu
\tableofcontents
\newpage


%%Seznam symbolů a zkratek%%
\phantomsection %Pro správné zobrazení čísla stránky v pdf rejsříku a přesný hyperodkaz 
\addcontentsline{toc}{section}{Seznam symbolů a zkratek}
\section*{Seznam symbolů a zkratek}

\begin{tabular}{ll}
\textbf{TE}\dotfill & Transverzálně elektrický \\  
\textbf{TEM}~\ldots\ldots\ldots & Transverzálně elektromagnetický \\ 
\textbf{TM}\dotfill & Transverzálně magnetický \\ 
\end{tabular} 
\newpage

\setlength{\parindent}{5ex} 
\setlength{\parskip}{1ex}


%%Úvod%%
\section{Úvod}
Předkládaná diplomová práce má za cíl rozšířit program Agros2D, resp. jeho knihovnu Hermes, vyvíjený na katedře KTE o modul umožňující modelování vysokofrekvenčního pole. Práce také shrnuje obecné poznatky v oblasti vysokofrekvenčního elektromagnetického pole a pokládá teoretický základ pro následnou tvorbu modulu.

Posuzování toho, co je vysoká frekvence je subjektivně závislé na oboru, v kterém se pohybujeme. Pro člověka pracujícího se síťovou frekvencí (50 Hz) bude 1 kHz již frekvence vysoká. Kdežto pro někoho navrhujícího satelitní spoje bude i 1 GHz frekvence nízká. Tato práce bude považovat za vysokofrekvenční zařízení systémy pracující s frekvencí odpovídající rádiovému a mikrovlnnému spektru - tj. 100 MHz až 1000 GHz. \cite{Pozar4}

\newpage

\section{Šíření vysokofrekvenčního elektromagnetického pole}
Každé nestacionární elektromagnetické pole má charakter elektromagnetické vlny, která se šíří prostředím. Toto pole lze popsat veličinami vztahujícími se k jednotlivému elektrickému a magnetickému poli:
\begin{table}[h] %tabulka "tabular" musí být uzavřena v plovoucím objektu "table"
\begin{center} %pro zarovnání tabulky na střed
\begin{tabular}{l l l}
$\vec{E}$~~~ & $V \cdot m^{-1}$~~~ & intenzita elektrického pole, \\ 
$\vec{D}$ & $C \cdot m^{-2}$ & elektrická indukce, \\ 
$\vec{H}$ & $A \cdot m^{-1}$ & intenzita magnetického pole, \\ 
$\vec{B}$ & $T$ & magnetická indukce. \\ 
\end{tabular} 
\end{center}
\end{table}

Vztahy mezi jednotlivými veličinami lze vyjádřit pomocí základních zákonů pro teorii elektromagnetického pole - soustavy Maxwellových rovnic, níže uvedených v diferenciálním tvaru:
\begin{subequations}
\begin{align}
\label{maxdif1}
\rot \vec{H} &= \vec{J} + \frac{\partial \vec{D}}{\partial t},
\\
\label{maxdif2}
\rot \vec{E} &= - \frac{\partial \vec{B}}{\partial t},
\\
\label{maxdif3}
\udiv \vec{D} &= \rho,
\\
\label{maxdif4}
\udiv \vec{B} &= 0,
\end{align}
\end{subequations}
kde $\vec{J} ~[A \cdot m^{-2}]$ je vektor proudové hustoty a $\rho ~[C \cdot m^{-3}]$ objemová hustota náboje. 

Vztahy mezi vektorem indukce ($\vec{D}$, $\vec{B}$) a vektorem intenzity pole ($\vec{E}$, $\vec{H}$) lze potom vyjádřit:
\begin{subequations}
\begin{align}
\label{epsilonE}
\vec{D} &= \varepsilon \vec{E} = \varepsilon _0 \varepsilon _r \vec{E},
\\
\label{muH}
\vec{B} &= \mu \vec{H} = \mu _0 \mu _r \vec{H},
\end{align}
\end{subequations}
kde $\varepsilon _0 ~[F \cdot m^{-1}]$ je permitivita vakua, $\varepsilon _r$ relativní permitivita prostředí nebo materiálu, $\mu _0 ~[H \cdot m^{-1}]$ permeabilita vakua a $\mu _r$ relativní permeabilita prostředí nebo materiálu.

Pro harmonicky proměnné pole lze Maxwellovy rovnice přepsat do frekvenční oblasti pomocí symbolicko-komplexní metody, nahrazením diference $\partial / \partial t$ za $\mj \omega$ a dosazením \ref{epsilonE} a \ref{muH}:
\begin{subequations}
\begin{align}
\rot \faz{H} &= \faz{J} + \mj \omega \varepsilon \faz{E},
\\
\rot \faz{E} &= - \mj \omega \mu \faz{H},
\\
\udiv (\varepsilon \faz{E}) &= \rho,
\\
\udiv (\mu \faz{H}) &= 0.
\end{align}
\end{subequations}

Výše uvedené rovnice platí v nepozměněné podobě pro lineární, homogenní, izotropní prostředí. Lineární znamená, že $\varepsilon , \mu$ a $\gamma$ nejsou závislé na veličinách pole, homogenní že $\varepsilon , \mu$ a $\gamma$ nejsou závislé na prostorových souřadnicích a izotropní že $\varepsilon , \mu$ a $\gamma$ nejsou závislé na směru vektorů veličin pole. $\gamma$ je měrná vodivost s jednotkou $S ^{-1}$.

\subsubsection*{Helmholtzova rovnice}
Pro harmonicky proměnné ($\frac{\partial}{\partial t} = \mj \omega$) pole bez vnějších zdrojů ($\rho = 0$, $\faz{J} _{ext} = 0$) mohou být první dvě Maxwellovy rovnice vyjádřeny ve formě fázoru:
\begin{subequations}
\label{maxvektor}
\begin{align}
\nabla \times \faz{E} &= -j\omega \mu \faz{H} ,
\\
\nabla \times \faz{H} &= j\omega \varepsilon \faz{E} .
\end{align}
\end{subequations}


Pomocí následujících kroků mohou být tyto rovnice upraveny do tvaru tzv. Helmholtzovy vlnové rovnice (pro fázor vektoru $E$):
\begin{subequations}
\label{helmholtz}
\begin{align}
\faz{H} &= \frac{\nabla \times \faz{E}}{-j \omega \mu}
\\
\nabla \times \frac{\nabla \times \faz{E}}{-j \omega \mu} &= j \omega \varepsilon \faz{E}
\\
\nabla \times \nabla \times \faz{E} &= \omega ^2 \mu \varepsilon \faz{E}
\\
k &= \omega \sqrt{\mu \varepsilon}
\\
\nabla \times \nabla \times \faz{E} &= k^2 \faz{E}
\\
\label{helmhomo}
- \nabla ^2 \faz{E} &= k^2 \faz{E}
\\
- \frac{\partial ^2}{\partial x^2} - \frac{\partial ^2}{\partial y^2} - \frac{\partial ^2}{\partial z^2} &= k^2 \faz{E}
\\
\left( \frac{\partial ^2}{\partial x^2} + \frac{\partial ^2}{\partial y^2} + \frac{\partial ^2}{\partial z^2} + k^2 \right) E_x &= 0 .
\end{align}
\end{subequations}

Konstanta $k ~[m ^{-1}]$ je definována jako konstanta šíření, nebo také  vlnové číslo. Odráží existenci materiálových parametrů prostředí, ve kterém se vlna šíří. Konstanta je závislá na permitivitě, permeabilitě a také na frekvenci:
\begin{equation}
k = \omega \sqrt{\mu \varepsilon} .
\end{equation}

Tvar uvedený v \ref{helmhomo} bývá nazýván jako \textbf{homogenní Helmholtzova rovnice} pro fázor vektoru $E$:
\begin{equation}
\nabla ^2 \faz{E} + k^2 \faz{E} = 0 ,
\end{equation}
a platí pro oblast bez vnějších zdrojů ($\rho = 0$, $\faz{J} _{ext} = 0$).

Pokud neplatí, že $\rho = 0$, $\faz{J} _{ext} = 0$, pravá strana rovnice není nulová (nejsou nulové členy obsahující $\rho$ a $\faz{J} _{ext}$) a tvar se označuje jako \textbf{nehomogenní Helmholtzova rovnice}:
\begin{equation}
\nabla ^2 \faz{E} + k^2 \faz{E} = \grad \frac{\rho}{\varepsilon} + \mj \omega \mu \faz{J} _{ext} .
\end{equation}

Podobně se postupuje i při výpočtu Helmholtzovy rovnice pro fázor vektoru $B$. Homogenní Helmholtzova rovnice má tvar:
\begin{equation}
\nabla ^2 \faz{B} + k ^2 \faz{B} = 0 .
\end{equation}
Rozšířením o pravou stranu v prostředí se vnějším zdroji ($\faz{J} _{ext} \neq 0$) dojdeme k nehomogenní Helmholtzově rovnici pro fázor vektoru $B$:
\begin{equation}
\nabla ^2 \faz{B} + k ^2 \faz{B} = - \mu \rot \faz{J} _{ext} .
\end{equation}

Zejména pro popis přenosových vedení se používá také konstanta $\gamma$ (nemá nic společného s gamou používanou pro veličinu konduktivity) vyjadřující měrnou vlnovou míru přenosu:
\begin{equation}
\gamma = \alpha + j \beta ,
\end{equation}
$\alpha ~[Np \cdot km ^{-1}]$ představuje měrný útlum a $\beta ~[rad \cdot km ^{-1}]$ měrný fázový posun.

Dalším parametrem, kterým je možné popsat určité prostředí, je vlnová impedance $Z ~[\Omega]$:
\begin{equation}
\faz{Z} = \frac{\omega \mu}{\faz{k}} = \sqrt{\dfrac{\mj \omega \mu}{\mj \omega \varepsilon + \omega}} .
\end{equation}
Vlnová impedance je obecně komplexní veličinou popisující poměr mezi intenzitou elektrického pole $\faz{E}$ a intenzitou magnetického pole $\faz{H}$:
\begin{equation}
\faz{Z} = \frac{\faz{E}}{\faz{H}} .
\end{equation}
Ve volném prostoru je vlnová impedance rovna tzv. vlastní (intrinsic) impedanci:
\begin{equation}
\faz{Z_0} = \eta _0 = \sqrt{\dfrac{\mu _0}{\epsilon _0}} = 377 \omega .
\end{equation}


\subsubsection*{Vlnová rovnice}
Popsat elektromagnetické vlnění lze velmi dobře pomocí vlnové rovnice odvoditelné z Maxwellových rovnic v diferenciálním tvaru. Použijeme \ref{maxdif1} a \ref{maxdif2}. Na levou stranu \ref{maxdif1} dosadíme \ref{muH}. Na pravou dosadíme $\rot{J} = \gamma \cdot \vec{E}$, $\gamma ~[S \cdot m^{-1}]$ a \ref{epsilonE}. V \ref{dosadit3max} potom na pravou stranu dosadíme \ref{maxdif3}
\begin{subequations}
\begin{align}
\frac{1}{\mu} \rot \vec{B} &= \mu \vec{E} + \varepsilon \frac{\partial \vec{E}}{\partial t}
\\
\rot \vec{B} &= \mu \gamma \vec{E} + \mu \varepsilon \frac{\partial \vec{E}}{\partial t}
\end{align}
\end{subequations}
\begin{subequations}
\begin{align}
\rot \vec{E} &= - \frac{\partial \vec{B}}{\partial t}
\\
\rot \rot \vec{E} &= \grad \udiv \vec{E} - \triangle \vec{E}
\\
- \frac{\partial \rot \vec{B}}{\partial t} &= \grad \udiv \vec{E} - \triangle \vec{E}
\\
- \frac{\partial \left( \mu \gamma \vec{E} + \mu \varepsilon \dfrac{\partial \vec{E}}{\partial t} \right) }{\partial t} &= \grad \udiv \vec{E} - \triangle \vec{E}
\\
\label{dosadit3max}
- \mu \gamma \frac{\partial \vec{E}}{\partial t} - \mu \varepsilon \frac{\partial ^2 \vec{E}}{\partial t ^2} &= \grad \udiv \vec{E} - \triangle \vec{E}
\\
\triangle \vec{E} - \mu \gamma \frac{\partial \vec{E}}{\partial t} - \mu \varepsilon \frac{\partial ^2 \vec{E}}{\partial t ^2} &= 0.
\end{align}
\end{subequations}

Podobně postupujeme i při výpočtu vektoru $H$, pouze místo \ref{maxdif3} použijeme \ref{maxdif4}. Výsledkem jsou tzv. \textbf{zobecněné vlnové rovnice}:
\begin{subequations}
\label{vlnovka}
\begin{align}
\triangle \vec{E} - \mu \gamma \frac{\partial \vec{E}}{\partial t} - \mu \varepsilon \frac{\partial ^2 \vec{E}}{\partial t ^2} &= 0,
\\
\triangle \vec{H} - \mu \gamma \frac{\partial \vec{H}}{\partial t} - \mu \varepsilon \frac{\partial ^2 \vec{H}}{\partial t ^2} &= 0.
\end{align}
\end{subequations}

Pokud na prostředí působí navíc proud vynucený vnější zdrojem ($\vec{J} _{\mathrm{ext}}$), přecházejí tyto rovnice do nehomogenní vlnové rovnice (pravá strana rovnic není rovna nule):
\begin{subequations}
\begin{align}
\triangle \vec{E} - \mu \gamma \frac{\partial \vec{E}}{\partial t} - \mu \varepsilon \frac{\partial ^2 \vec{E}}{\partial t ^2} &= \grad \frac{\rho}{\varepsilon} + \mu \frac{\partial \vec{J} _{\mathrm{ext}}}{\partial t} ,
\\
\triangle \vec{H} - \mu \gamma \frac{\partial \vec{H}}{\partial t} - \mu \varepsilon \frac{\partial ^2 \vec{H}}{\partial t ^2} &= - \mu \rot \vec{J} _{\mathrm{ext}}.
\end{align}
\end{subequations}

Téměř identické rovnice platí také pro vektory elektrické indukce $\vec{D}$ a magnetické indukce $\vec{B}$:
\begin{subequations}
\label{DHnehom}
\begin{align}
\triangle \vec{D} - \mu \gamma \frac{\partial \vec{D}}{\partial t} - \mu \varepsilon \frac{\partial ^2 \vec{D}}{\partial t ^2} &= \grad \frac{\rho}{\varepsilon} + \mu \frac{\partial \vec{J} _{\mathrm{ext}}}{\partial t} ,
\\
\triangle \vec{B} - \mu \gamma \frac{\partial \vec{B}}{\partial t} - \mu \varepsilon \frac{\partial ^2 \vec{B}}{\partial t ^2} &= - \mu \rot \vec{J} _{\mathrm{ext}}.
\end{align}
\end{subequations}
Výše uvedené nehomogenní rovnice \ref{DHnehom} přejdou při $\vec{J} _{\mathrm{ext}} = 0$ do rovnic homogenních stejně jako \ref{vlnovka} (pravá strana rovnic bude rovna 0).



\newpage
\subsection{Šíření vln pomocí vlnovodů a vedení}
\subsubsection{Vlnovody}
Píše se rok 1893 a britský matematik a fyzik Oliver Heaviside pracuje na jednom ze svých projektů. Zabývá se možností šíření elektromagnetických vln uvnitř duté trubice. A tuto myšlenku odmítá. Věří, že elektromagnetickou energii nelze přenášet bez pomoci dvou vodičů. O čtyři roky později, roku 1897, tuto otázku přezkoumává jiný britský fyzik, Lord Reyleigh. A matematicky dokazuje, že šíření elektromagnetických vln dutou trubicí je možné, a to jak pro vlnovody s kruhovým, tak pravoúhlým průřezem.

\begin{figure}[h] 
\begin{center}
\includegraphics[width=8cm]{vlnovod.pdf}
\caption{Vlnovod masivnější konstrukce pro šíření vln o nižší frekvenci (okolo 1 GHz)}
\label{vlnovod}
\end{center}
\end{figure}

Výhodou vlnovodů je jejich nízký útlum a schopnost přenášet vysoké výkony, stinnou stránkou ale je jejich rozměrnost (čím nižší frekvence, tím masivnější konstrukce) a cena. Pro frekvenci kolem 1 GHz jsou rozměry již 20 cm x 10 cm a s klesající frekvencí dále rostou (viz. obr. \ref{vlnovod}. Pro vysoké frekvence (nad 100 GHz) naopak rozměry klesají na desetiny milimetru a takovéto vlnovody jsou proto náročné na výrobu. Nezřídka se proto používají tzv. nadrozměrné vlnovody, kdy velikost vlnovodu překročí hranici pro šíření jednoho vidu a vlnovodem se šíří vidů více. 
%Možno doplnit: "Pro dané rozměry a materiálové vlastnosti vlnovodu se tato hraniční frekvence pro šíření jednoho vidu nazývá frevencí mezní (cutoff) a bude o ní řeč později"

Druhým způsobem pro přenos vysokých frekvencí jsou koaxiální vedení. Koaxiální vedení vynikají šířkou pásma, kterou mohou přenášet a jsou i cenově dostupnější, nicméně již méně jsou vhodná pro stavbu komplexních mikrovlnných komponent.

Alternativu představují planární vlnovody vyráběné ve formě stripline, microstrip, koplanárního vlnovodu a mnoha dalších. Kromě malých rozměrů a nízké ceny umožňují díky planární technologii výroby i snadnou integrovatelnost s dalšími elektronickými prvky a stavbu mikrovlnných obvodů.

Microstrip je dnes nejpoužívanějším médiem pro integrované mikrovlnné obvody. Sestává se z vodivého proužkou, který je od země oddělen dielektrickým substrátem, jak je znázorněno na obr. \ref{microstrip}. Microstrip byl vyvinut v laboratořích ITT jako konkurent jiné technologii, zveřejněné v roce 1952, stripline. Nevýhodou technologie microstrip v porovnání s klasickými vlnovody jsou její vyšší ztráty a nižší výkon, který je schopna přenášet. Z důvodu neuzavřenosti je také microstrip náchylnější na přeslechy a rušení.

\begin{figure}[h] 
\begin{center}
\includegraphics[width=5cm]{microstrip.png}
\caption{Microstrip a jeho části}
\label{microstrip}
\end{center}
\end{figure}

Elektromagnetická vlna šířící se microstripem existuje z části v dielektrickém substrátu a z části ve vzduchu nad ním. A jelikož dielektrická konstanta substrátu je jiná (větší) než vzduchu, vzniká nehomogenní médium, skrz které se vlna šíří. Nehomogenita a frekvenční rozptyl se ještě dále zhoršují spolu s větší šířkou substrátu. Proto až zvládnutí technologie výroby dostatečně tenkých substrátů umožňující menší frekvenční závislost vedení a potlačení podélných složek elektromagnetického pole umožnilo široké rozšíření a skutečný nástup technologie microstrip.
%Možno doplnit: Microstrip neumožňuje šíření skutečných TEM vln. Při nenulové frekvenci jsou vždy přítomné i podélné složky $\vec{E}$ a $\vec{H}$ pole. Tyto složky jsou ale zanedbatelně malé, proto lze o přítomném poli hovořit jako o kvazi-TEM.

\subsubsection{Typy vln šířících se prostřednictvím vlnovodů}
Za běžných okolností nabývají vektory $\vec{E}$ a $\vec{H}$ hodnot v osách roviny kolmé na šíření pole i roviny rovnoběžné s šířením pole (podélné). V určitých případech však může dojít k potlačení podélné složky šíření. Přenosová vedení skládající se ze dvou a více vodičů umožňují šíření tzv. TEM vln (Transverse ElectroMagnetic - transverzálně elektromagnetických). Tyto vlny postrádají podélnou složku elektrického i magnetického pole. Vlnovody sestávající se z jednoho vodiče umožňují šíření TE vln (Transverse Electric - transverzálně elektrických) postrádajících podélnou složku elektrického pole nebo TM vln (Transverse magnetic - transverzálně magnetických) postrádajících podélnou složku magnetického pole. Nikoli však obou současně. Výhodou TEM vln je, že mají jedinečně definované napětí, proud a charakteristickou impedanci. U TE a TM vln toto jednoznačné určení charakteristické impedance není možné, ale existuje matematický postup, který dokáže úspěšně pracovat s modelem charakteristické impedance i u těchto vln.

\begin{figure} 
\begin{center}
\includegraphics{geometrie.png}
\caption{Geometrie válcového vlnovodu v kartézském souřadném systému}
\label{geometrie}
\end{center}
\end{figure}

\subsubsection{Maxwellovy rovnice pro šíření uvnitř válcových vlnovodů}
Předpokladem je šíření harmonického pole se závislostí $\me ^{jwt}$ podél osy $z$ válcového přenosového vedení nebo vlnovodu, které je ve směru osy $z$ uniformní, nekonečně dlouhé a dokonale vodivé. Geometrie takového vlnovodu je znázorněna na obr. \ref{geometrie} Elektrické a magnetické pole potom může být vyjádřeno takto:
\begin{subequations}
\label{hromadna}
\begin{align}
\faz{E}(x, y, z)&=[\faz{e}(x, y) + \hat{z}e_z(x, y)] \me ^{j\beta z} ,
\\
\faz{H}(x, y, z)&=[\faz{h}(x, y) + \hat{z}h_z(x, y)] \me ^{j\beta z} .
\end{align}
\end{subequations}
$\vec{e}(x,y)$ a $\vec{h}(x,y)$ vyjadřují příčnou (transverzální) složku elektrického a magnetického pole, $e_z$ a $h_z$ představují složku podélnou. Pokud by se vlna šířila v opačném směru (ne ve směru $+z$, ale $-z$), nahradí se $\beta$ za $-\beta$. V případě, kdy by byl přítomen útlum, nahradí se se konstanta šíření $j \beta$ za komplexní konstantu $\gamma$, $\gamma = \alpha + j \beta$.

Pokud vlnovod nebo přenosové vedení jsou nezřídlové mohou být Maxwellovy rovnice vyjádřeny:
\begin{subequations}
\label{maxvektor2}
\begin{align}
\nabla \times \faz{E} &= -j\omega \mu \faz{H} ,
\\
\nabla \times \faz{H} &= j\omega \varepsilon \faz{E} .
\end{align}
\end{subequations}

Tyto vektorové rovnice mohou být převedeny do tvaru jednotlivých parciálních diferenciálních rovnic:
\begin{subequations}
\begin{align}
\frac{\partial E_z}{\partial y} + j \beta E_y &= -j \omega \mu H_x ,
\label{maxpart1}
\\
-j \beta E_x - \frac{\partial E_z}{\partial x} &= - j \omega \mu H_y ,
\\
\frac{\partial E_y}{\partial x} - \frac{\partial E_x}{\partial y} &= -j \omega \mu H_z ,
\end{align}
\end{subequations}
\begin{subequations}
\begin{align}
\frac{\partial H_z}{\partial y} + j \beta H_y &= j \omega \varepsilon E_x ,
\label{maxpart4}
\\
-j \beta H_x - \frac{\partial H_z}{\partial x} &= j \omega \varepsilon E_y ,
\label{maxpart5}
\\
\frac{\partial H_y}{\partial x} - \frac{\partial H_x}{\partial y} &= j \omega \varepsilon E_z .
\end{align}
\end{subequations}

Z těchto rovnic mohou být následně vyjádřeny příčné (transverzální) složky jednotlivých $E$ a $H$ polí:
\begin{subequations}
\label{vyjadreno}
\begin{align}
H_x &= \frac{j}{k^2_c} \left( \omega \varepsilon \frac{\partial E_z}{\partial y} - \beta \frac{\partial H_z}{\partial x} \right) ,
\\
H_y &= \frac{-j}{k^2_c} \left( \omega \varepsilon \frac{\partial E_z}{\partial x} + \beta \frac{\partial H_z}{\partial y} \right) ,
\\
E_x &= \frac{-j}{k^2_c} \left( \beta \frac{\partial E_z}{\partial x} + \omega \mu \frac{\partial H_z}{\partial y} \right) ,
\\
E_y &= \frac{j}{k^2_c} \left(- \beta \frac{\partial E_z}{\partial y} + \omega \mu \frac{\partial H_z}{\partial x} \right) .
\end{align}
\end{subequations}

Příklad výpočtu pro získání složky $H_x$ dosazením $E_y$ z \ref{maxpart5} do \ref{maxpart1}:
%\begin{subequations}
%momentálně uděláno jako nečíslované (pro výpočty)
\begin{align*}
%dále uvažuji, zda centrovat "=" nebo ne
%1. řádek (a)
E_y &= - \frac{\beta H_x}{\omega \varepsilon} - \frac{1}{j \omega \varepsilon} \frac{\partial H_z}{\partial x}
\\
%2. řádek (b)
\frac{\partial E_z}{\partial y} + j \beta \left( - \frac{\beta H_x}{\omega \varepsilon} - \frac{1}{j \omega \varepsilon} \frac{\partial H_z}{\partial x} \right) &= -j \omega \mu H_x
\\
%3. řádek (c)
\frac{\partial E_z}{\partial y} - j \frac{\beta ^2 H_x}{\omega \varepsilon} - \frac{\beta}{\omega \varepsilon} \frac{\partial H_z}{\partial x} &= -j \omega \mu H_x
\\
%4. řádek (d)
\frac{\partial E_z}{\partial y} - \frac{\beta}{\omega \varepsilon} \frac{\partial H_z}{\partial x} &= -j \omega \mu H_x + j \frac{\beta ^2 H_x}{\omega \varepsilon}
\\
%5. řádek (e)
H_x &= \frac{\dfrac{\partial E_z}{\partial y} - \dfrac{\beta ^2}{\omega \varepsilon} \dfrac{\partial H_z}{\partial x}}{-j \omega \mu + j \dfrac{\beta ^2}{\omega \varepsilon}}
\\
%6. řádek (f)
H_x &= j \frac{\dfrac{\partial E_z}{\partial y} - \dfrac{\beta ^2}{\omega \varepsilon} \dfrac{\partial H_z}{\partial x}}{\dfrac{\omega ^2 \varepsilon \mu - \beta ^2}{\omega \varepsilon}}
\\
%7. řádek (g)
H_x &= j \left( \frac{\partial E_z}{\partial y} - \frac{\beta}{\omega \varepsilon} \frac{\partial H_z}{\partial x} \right) \frac{\omega \varepsilon}{\omega ^2 \varepsilon \mu - \beta ^2}
\\
%8. řádek (h)
k^2_c &= k^2 - \beta ^2 = {(\omega \sqrt{\mu \varepsilon})}^2 - \beta ^2 = \omega ^2 \mu \varepsilon - \beta ^2
\\
%9. řádek (i)
H_x &= \frac{j}{k^2_c} \left( \mu \varepsilon \frac{\partial E_z}{\partial y} - \beta \frac{\partial H_z}{\partial x} \right),
\end{align*}
%\end{subequations}
%nevím, jak správně vyjádřit "cutoff" wavenumber - vlnové číslo ?odříznutí? ?mezní?
$k$ je vlnové číslo materiálu uvnitř přenosového vedení nebo vlnovodu, $k_c$ potom tzv. mezní (cutoff) vlnové číslo. V případě přítomnosti ztrát je $\varepsilon$ nahrazeno komplexním $\varepsilon (1 - j \tan{\gamma})$, kde $\tan{\gamma}$ představuje ztrátový tangens materiálu.

\subsubsection{TEM vlny}
Pro TEM (transverzálně elektromagnetické) vlny platí, že $E_z = 0$ a $H_z = 0$ (není zde přítomna podélná složka elektrického ani magnetického pole). Dosazením těchto hodnot do \ref{vyjadreno} zjistíme, že $H, E = j (0-0) / k^2_c$ a tedy $H, E = 0 / (k^2_c)$. Příčné složky polí jsou tedy také nulové. Pokud se ovšem $k^2_c = 0$, potom získáváme neurčitý výsledek. To mimo jiné znamená, že mezní (cutoff) vlnové číslo pro TEM vlny $k_c = 0$.

Pokud dosadíme $E_z = 0$ do \ref{maxpart1}, $H_x$ poté do \ref{maxpart5} a $H_z = 0$, získáme:
%\begin{subequations}
%momentálně uděláno jako nečíslované (pro výpočty)
\begin{align*}
j \beta E_y &= -j \omega \mu H_x
\\
H_x &= - \frac{\beta E_y}{\omega \mu}
\\
-j \beta \left( - \frac{\beta E_y}{\omega \mu} \right) &= j \omega \varepsilon E_y
\\
\frac{\beta ^2}{\omega \mu} &= \omega \varepsilon
\\
\beta ^2 &= \omega ^2 \mu \varepsilon
\\
\beta &= \omega \sqrt{\mu \varepsilon} .
\end{align*}
%\end{subequations}

%V \ref{maxvektor} byly uvedeny Maxwellovy rovnice ve vektorovém tvaru pro nezřídlové elektrické a magnetické pole. Tyto rovnice mohou být dále upraveny do tzv. Helmholtzovy vlnové rovnice:
%\begin{subequations}
%\label{helmholtz}
%\begin{align}
%\vec{H} &= \frac{\nabla \times \vec{E}}{-j \omega \mu}
%\\
%\nabla \times \frac{\nabla \times \vec{E}}{-j \omega \mu} &= j \omega \varepsilon \vec{E}
%\\
%\nabla \times \nabla \times \vec{E} &= \omega ^2 \mu \varepsilon \vec{E}
%\\
%k &= \omega \sqrt{\mu \varepsilon}
%\\
%\nabla \times \nabla \times \vec{E} &= k^2 \vec{E}
%\\
%- \nabla ^2 \vec{E} &= k^2 \vec{E}
%\\
%- \frac{\partial ^2}{\partial x^2} - \frac{\partial ^2}{\partial y^2} - \frac{\partial ^2}{\partial z^2} &= k^2 \vec{E}
%\\
%\left( \frac{\partial ^2}{\partial x^2} + \frac{\partial ^2}{\partial y^2} + \frac{\partial ^2}{\partial z^2} + k^2 \right) E_x &= 0 .
%\end{align}
%\end{subequations}

V \ref{helmholtz} bylo uvedeno odvození Helmholtzovy rovnice. Její tvar pro $k = 0$ a příčné pole bez přítomnosti podélné složky $z$ lze vyjádřit:
\begin{subequations}
\begin{align}
\left( \frac{\partial ^2}{\partial x^2} + \frac{\partial ^2}{\partial y^2} \right) E_x &= 0 ,
\\
\left( \frac{\partial ^2}{\partial x^2} + \frac{\partial ^2}{\partial y^2} \right) E_y &= 0 .
\end{align}
\end{subequations}

Tyto rovnice potom mohou být přepsány do formy rovnic uvedených v \ref{hromadna}
\begin{equation}
\nabla ^2_t \faz{e} (x, y) = 0 ,
\end{equation}
využívající Laplacova operátoru ve dvourozměrném příčného pole.

Stejný postup lze aplikovat i u příčného magnetického pole a dojít ke stejnému výsledku:
\begin{equation}
\nabla ^2_t \faz{h} (x, y) = 0 .
\end{equation}

Z toho plyne, že příčné pole TEM vlny je stejné jako statické pole mezi dvěma vodiči. A v případě elektrostatiky lze elektrické pole vyjádřit jako gradient skalárního potenciálu:
\begin{equation}
\faz{e} (x, y) = - \nabla _t \Phi (x, y) .
\end{equation}
Aby tato rovnice ale mohla platit, musí být ještě splněna podmínka, že pole musí být nevírové:
\begin{equation}
\nabla _t \times \faz{e} = 0 ,
\end{equation}
což v tomto případě opravdu platí:
\begin{equation}
\nabla _t \times \faz{e} = - j \omega \mu h_z \hat{z} = 0 .
\end{equation}

I toto pole lze potom tedy vyjádřit jako Laplacovu rovnici:
\begin{equation}
\nabla ^2_t \Phi (x,y) = 0 .
\end{equation}

V elektrostatice lze napětí vyjádřit jako rozdíl dvou potenciálů:
\begin{equation}
\label{napeti}
U_{12} = \phi _1 - \phi _2 = \int ^2_1 \faz{E} \ud \faz{l} .
\end{equation}
Proud se poté získá z Ampérova zákona:
\begin{equation}
\label{amper}
I = \oint _C \faz{H} \ud \faz{l}.
\end{equation}

Vlnová impedance TEM vlny se získá jako podíl příčného elektrického a magnetického pole. První rovnice vychází z \ref{maxpart4} při $\beta = \omega \sqrt{\mu \varepsilon}$:
\begin{subequations}
\label{ztem1}
\begin{align}
Z_{TEM} &= \frac{E_x}{H_y}
\\
j \beta H_y &= j \omega \varepsilon E_x
\\
\frac{\beta}{\omega \varepsilon} &= \frac{E_x}{H_y}
\\
\frac{E_x}{H_y} &= \frac{\omega \sqrt{\mu \varepsilon}}{\omega \varepsilon} = \sqrt{\dfrac{\mu}{\varepsilon}} = \eta .
\end{align}
\end{subequations}
Druhá vychází z \ref{maxpart1}:
\begin{subequations}
\label{ztem2}
\begin{align}
Z_{TEM} &= - \frac{E_y}{H_x}
\\
j \beta E_x &= - j \omega \mu H_y
\\
\frac{E_y}{H_x} &= - \frac{\omega \mu}{\beta}
\\
- \frac{E_y}{H_x} &= \frac{\omega \mu}{\omega \sqrt{\mu \varepsilon}} = \sqrt{\dfrac{\mu}{\varepsilon}} = \eta .
\end{align}
\end{subequations}
Výsledky z \ref{ztem1} a \ref{ztem2} je možno spojit do jedné rovnice:
\begin{equation}
\faz{h} (x, y) = \frac{1}{Z_{TEM}} \hat{z} \times \faz{e} (x, y) .
\end{equation}

Charakteristická impedance $Z_0$ se vypočte jako $U / I$ a lze jí získat z \ref{napeti} a \ref{amper}

\subsubsection{TE vlny}
Na rozdíl od TEM vln, kde jsou podélné složky $H$ i $E$ pole se u TE vln $H_z  \neq 0$. Pro přítomnost podélné složky magnetického pole se jim také říka $H$-vlny. U TE vln mohou být rovnice \ref{vyjadreno} zjednodušeny na:
\begin{subequations}
\begin{align}
H_x &= \frac{-j \beta}{k^2_c} \frac{\partial H_z}{\partial x} ,
\\
H_y &= \frac{-j \beta}{k^2_c} \frac{\partial H_z}{\partial y} ,
\\
E_x &= \frac{-j \omega \mu}{k^2_c} \frac{\partial H_z}{\partial y} ,
\\
E_y &= \frac{j \omega \mu}{k^2_c} \frac{\partial H_z}{\partial x} .
\end{align}
\end{subequations}

U TE vln neplatí, že $k_c = 0$, a konstanta šíření $\beta$ je funkcí frekvence a geometrie vedení. Nejprve se proto musí spočítat $H_z$ z Helmoholtzovi vlnové rovnice (odvození této rovnice je nastíněno v \ref{helmholtz}):
\begin{equation}
\left( \frac{\partial ^2}{\partial x^2} + \frac{\partial ^2}{\partial y^2} + \frac{\partial ^2}{\partial z^2} + k^2 \right) H_z = 0 .
\end{equation}
Tato rovnice může být pro $H_z(x,y,z) = h_z(x,y)e^{-j \beta z}$ upravena na:
\begin{equation}
\left( \frac{\partial ^2}{\partial x^2} + \frac{\partial ^2}{\partial y^2} + k^2_c \right) h_z = 0 .
\end{equation}
K vyřešení této rovnice musí být tedy známy hraniční podmínky a geometrie konkrétního vlnovodu.

Impedance TE vlny je:
\begin{subequations}
\begin{align}
Z_{TE} &= - \frac{E_y}{H_x}
\\
j \beta E_y &= - j \omega \mu H_x
\\
- \frac{E_y}{H_x} &= \frac{\omega \mu}{\beta} ,
\end{align}
\end{subequations}
z čehož plyne, že $Z_{TE}$ je frekvenčně závislé.

\subsubsection{TM vlny}
TM vlny jsou podobné TE vlnám tím, že je v nich též jedna podélná složka pole přítomna. V tomto případě je to složka $z$ pole $H$. Platí tedy $E_z = 0$, $H_z \neq 0$. Tyto vlny jsou proto také někdy nazývány jako $H$ vlny. Rovnice \ref{vyjadreno} mohou být pro TM vlny vyjádřeny jako:
\begin{subequations}
\begin{align}
H_x &= \frac{j \omega \varepsilon}{k^2_c} \frac{\partial E_z}{\partial y} ,
\\
H_y &= \frac{-j \omega \varepsilon}{k^2_c} \frac{\partial E_z}{\partial x} ,
\\
E_x &= \frac{-j \beta}{k^2_c} \frac{\partial E_z}{\partial x} ,
\\
E_y &= \frac{-j \beta}{k^2_c} \frac{\partial E_z}{\partial y} .
\end{align}
\end{subequations}

Stejně jako u TE vln, neplatí, že $k_c = 0$, a konstanta šíření $\beta$ je funkcí frekvence a geometrie vedení. Nejprve se proto musí spočítat $E_z$ z Helmoholtzovi vlnové rovnice (odvození této rovnice je nastíněno v \ref{helmholtz}):
\begin{equation}
\left( \frac{\partial ^2}{\partial x^2} + \frac{\partial ^2}{\partial y^2} + \frac{\partial ^2}{\partial z^2} + k^2 \right) E_z = 0 .
\end{equation}
Tato rovnice může být pro $E_z(x,y,z) = e_z(x,y)e^{-j \beta z}$ upravena na:
\begin{equation}
\left( \frac{\partial ^2}{\partial x^2} + \frac{\partial ^2}{\partial y^2} + k^2_c \right) e_z = 0 .
\end{equation}
K vyřešení této rovnice musí být tedy známy hraniční podmínky a geometrie konkrétního vlnovodu.

Impedance TM vlny je:
\begin{subequations}
\begin{align}
Z_{TM} &= \frac{E_x}{H_y}
\\
j \beta H_y &= j \omega \varepsilon E_x
\\
\frac{E_x}{H_y} &= \frac{\beta}{\omega \varepsilon} ,
\end{align}
\end{subequations}
z čehož plyne, že $Z_{TM}$ je frekvenčně závislé.



%%Závěr%%
\clearpage %\newpage nefunguje, po plouvocím objektu (obrázek, tabulka) musí pro novou stránku následovat "clearpage" 
\section{Závěr}
Věřím, že se povede dojít až k úspěšnému odvození finálních rovnic a vytvoření modulu pro modelování vysokofrekvenčního pole.

V současné době zdokonaluji kapitolu o vlnovodech a chtěl bych odvodit okrajové podmínky pro Helmholtzovu rovnici (ale nevím, co to je ani jak na to) - samotnou Helmholtzovu rovnici jsem již odvodil. Také bych Helmholtzovu rovnici chtěl vyjádřit ve válcových souřadnicích.

V další fázi bych dotáhl do konce kapitolu o šíření vln ve volném prostoru a vytvořil kapitolu (nebo přílohu) se stručnými informacemi o programu Agros2D a jeho ovládání.

Poté přijde na řadu to nejtěžší. Zjistit bližší informace o tom, co jsou to tzv. slabé formy, odvodit rovnice, vytvořit modul a analyzovat jeho chování uvnitř programu Agros2D.

Na závěr potom porovnat jeho chování a modelování vysokofrekvenčních zařízení v programu Comsol.

%%Použitá literatura%%
\newpage
\phantomsection %Pro správné zobrazení čísla stránky v pdf rejsříku a přesný hyperodkaz
\addcontentsline{toc}{section}{Použitá literatura}
\begin{thebibliography}{10}

%Do "{}" příjmení autora, za to vložit z citace.cz, na název kurzívu
%Podle názvu v "{}" se poté odkazuje uvnitř textu pomocí příkazu "\cite{}" 
\bibitem{Pozar2} POZAR, David M. \textit{Microwave Engineering}. John Wiley \& Sons, Inc., 1998. Second Edition. ISBN 0-471-17096-8.
\bibitem{Pozar4} POZAR, David M. \textit{Microwave Engineering}. John Wiley \& Sons, Inc., 2012. Fourth Edition. ISBN 978-0-470-63155-3.
\bibitem{Kastner} KASTNER, Raphael. SCHOOL OF ELECTRICAL ENGINEERING, Tel Aviv University. Antennas and Radiation. Tel Aviv, 2009. Dostupné z: http://www.scribd.com/doc/56316827/3/Maxwell-Equations-in-the-Frequency-Domain
\bibitem{Dimensions} Rectangular waveguide dimensions. Microwaves101.com [online]. Tucson, AZ, November 3, 2012 [cit. 2012-11-19]. Dostupné z: http://www.microwaves101.com/encyclopedia/waveguidedimensions.cfm
\bibitem{Microstrip} Microstrip. Microwaves101.com [online]. Tucson, AZ, February 18, 2012 [cit. 2012-11-19]. Dostupné z: http://www.microwaves101.com/encyclopedia/microstrip.cfm
\bibitem{Instrinsic} Intrinsic Impedance. Antenna-Theory.com [online]. (c) 2009-2011 [cit. 2012-11-19]. Dostupné z: http://www.antenna-theory.com/definitions/intrinsicimpedance.php
\bibitem{Agros} KARBAN, Pavel. FACULTY OF ELECTRICAL ENGINEERING, University of West Bohemia in Pilsen. Agros2D [online]. Plzeň, 2012 [cit. 2012-11-19]. Dostupné z: http://www.agros2d.org/


\end{thebibliography}

%%Seznam obrázků%%
\newpage
\phantomsection %Pro správné zobrazení čísla stránky v pdf rejsříku a přesný hyperodkaz
\addcontentsline{toc}{section}{Seznam obrázků}
\setlength{\parskip}{0ex}%Aby nebyly moc velké mezery mezi řádky seznamu obrázků a tabulek
\listoffigures


%%Přílohy%%
\newpage
\appendix %deklarace, že se jedná o přílohu

\section{Harmonogram činností}
\begin{table}[h]
\begin{tabular}{|l|l|}
\hline 
\textbf{~~~Měsíc a rok~~~} & \textbf{~~~~~~~~~~~~~~~Rozpis činností~~~~~~~~~~~~~~~} \\ 
\hline 
Listopad 2012 & Helmhlotzova rovnice, vlnovody \\ 
\hline 
Prosinec 2012 & Šíření vln ve volném prostoru, Agros2D \\ 
\hline 
Leden 2013 & Metoda konečných prvků \\ 
\hline 
Únor 2013 & Návrh modulu pro VF pole \\ 
\hline 
Březen 2013 & Ověření funkčnosti proti Comsolu, ladění \\ 
\hline 
Duben 2013 & Opravy, dodělávky \\ 
\hline 
\end{tabular} 
\end{table}

\vspace*{20mm}
\begin{flushright}
Vedoucí práce: Ing. David Pánek, Ph.D.
\hspace{10mm}

\vspace*{10mm}
Podpis vedoucího práce: \ldots \ldots \ldots \ldots \ldots \ldots \ldots \ldots \ldots \ldots

\vspace*{10mm}
Datum: \ldots \ldots \ldots \ldots \ldots \ldots \ldots \ldots \ldots \ldots
\end{flushright}

\end{document}
